 
%Covering letter concerning your manuscript

%Please paste or type in your cover letter explaining why we should publish your manuscript and elaborating on any issues relating to our editorial policies detailed in the instructions for authors,
% and declaring any potential competing interests. 

We present a highly flexible, versatile, and custom-made instrumentation system, called EUDET-type beam telescope, that enables researches to test and characterise newly developed particle detectors.
Making use of custom-developed sensors, it allows for the reconstruction of particle trajectories down to few micrometres. 
This enables the researcher to scan their detectors with unprecedented spatial resolution in comparison to other beam telescopes. 
A unique feature is the easy integrability of user data acquisition systems into the framework enabling a wide community to make use of the instrument. 
With a growing need of the sensor R\&D community, several copies of this instrument have already been built and are used around the world. 
This publication should serve as the reference publication to be cited for all users publishing results derived from data acquired with a EUDET-type beam telescope.
The submitted article represents part of the work conducted within the 6th Framework Programme “Structuring the European Research Area” (EUDET) supported by the European Commission. 

For all pictures and figures we grant an irrevocable non-exclusive simple usage right to the publisher in order to publish and distribute them in all forms and media.


% 
% The data acquisition system of the particle detector under test is easily integrable into the framework on both the software and the hardware level, which makes the EUDET-type beam telescopes unique
%  in comparison to other beam telescopes. 
% All parts of the system are described: The data acquisition system, the software framework for data taking and the offline data analyses software. 
% 
% Innovative software for data tracking and offline data analyses has been developed to complement the hardware. 
% One copy of this instrument is used at DESY, Hamburg, Germany, two copies are at CERN, Geneva, Switzerland. 
% More copies are at synchrotrons at SLAC, Menlo Park, at FermiLab, Chicago and at the University of Bonn, Germany. 
% 
% 
% 
% 
%     Laboratory-specific custom instrumentation and diagnostics
%     Innovative and clever experimental techniques
%     Subtle refinements to established experimental setups
%     Novel analytical methods
