
The modular cross platform data acquisition framework \eudaq~\cite{ref:eudaqwebsite} has been designed and developed to serve as flexible and simple to use data taking software for the DESY-type beam telescopes,
 allowing for easy integration of other devices. 
It consists of completely independent modules communicating via TCP/IP enabling a distributed data acquisition with modules running on separate machines. 
Currently, \eudaq\ is designed for synchronous DAQ systems requiring one event per trigger per attached subdetector system before building the global event. 
Thus, the trigger rate is always limited by the slowest device.

The central interaction point for users with the framework is the Run Control and its graphical user interface (GUI). 
All other modules connect to the Run Control at startup and receive additional information from there during operation such as the commands for starting and stopping a DAQ run. 
The GUI provides all controls necessary to the user on shift. 
Another important user interface is the Log Collector, gathering logging information from all modules displaying them in one unified logging window. 
Log files are stored with recorded data files for later reference.

The actual detector data is delivered to the framework by so-called Producers.
Producers are the links between the EUDAQ framework and the subdetector systems such as the beam telescope, the TLU, or user DAQ systems.
They interface with the EUDAQ library and provide a set of commands to be called by the Run Control. 
This simple interface scheme eases the integration of user DAQ systems into the software framework.
The data read out from the detectors by the individual producers is sent to the so-called Data Collector. 
This Data Collector is repsonsible for the event building, i.e.\ the correlation of events from all subdetector systems to single global events comprising all data belonging to one trigger. 
Basic sanity checks such as the event numbers from the individual subdetectors are executed.

To ensure data quality during data acquisition the Online Monitor tool is avaliabe. 
It connects to the Data Collector requesting a fixed fraction of the recorded events (e.g.\ one out of a hundered) to fully decode all subdetector data
 and build basic plots such as hit maps or correlation plots.
Thereby, data quality monitoring verifying that the different devices are synchronized in time and all within the geometrical trigger acceptance, is rendered possible.

The data decoding is done using Data Converter plug-ins for every detector type attached to the Run Control. 
The plug-in to be called for a specific subevent is deducted from the event type transmitted by the Producer and written to the data stream by the Data Collector. 
Each Data Converter plug-in can implement several data format end points to allow e.g.\ the conversion to an internal EUDAQ format for the Online Monitor, to simple ROOT trees, or to LCIO
 which is used by the offline reconstruction software described in the following section.

Configuration of the data acquisition framework is performed via global configuration files. 
The information for every individual module is parsed and distributed by the Run Control. 
The configuration file is a plain text file divided into sections for the individual modules, which contain a list of parameter-value pairs defined by the module.
The full content of the configuration file including commented lines is stored in the so-called Begin-Of-Run Event (BORE) of every run and is thus available later for offline analysis and reference. 
This greatly simplifies book keeping of detector parameters during test beam shifts since all settings are stored automatically.
