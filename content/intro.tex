
%History of DESY beam telescopes, from Eudet to the present

For detector R\&D projects reaching from collider-specific detectors\footnote{e.g.~at high-luminosity LHC experiments\,\cite{Nurnberg:2014aya, Garcia-Argos:2015zda} with high radiation-hardness requirements,
 or at the International Linear Collider\,\cite{ILC}, the Compact Linear Collider \,\cite{CLIC} and Belle\,II\,\cite{Belle} with high resolution and low mass requirements\,\cite{MAPS}}, %FIXME what else?!
  to medical applications\footnote{The medipix chip \cite{medi}}, to HV-CMOS studies\,\cite{HV-CMOS} or generic detector R\&D studies,
  test beam facilities and suitable beam telescopes are a vital tools to get a detailed understanding of the physics taking place inside newly designed devices. 
Complementary to sensor simulations using finite element analysis tools, test beams studies are used at various stages during the development of either sensor themselves,
 or subsequently to test a complete demonstrator including a full data acquisition framework. 
%Also in future high-energy physics experiments pixel detectors will be used in the inner layers of the tracking devices. 
%The demands are challenging and range from high speed and high radiation-hardness for the high-luminosity LHC (HL-LHC)
% experiments\,\cite{Nurnberg:2014aya, Garcia-Argos:2015zda} to high resolution and low mass at the International Linear Collider (ILC)\,\cite{ILC} or at the Compact Linear Collider (CLIC)\,\cite{CLIC}. 
These studies are well suited and often used for the evaluation of the performance of a detector prototype. % developed within the various R\&D projects. 
In order to facilitate these measurements a high-resolution pixel beam telescope was developed within the $\eudet$ project\,\cite{ref:eudetreport200902},
 with the clear guideline to allow for the integration of user data acquisition systems for a wide range of read-out schemes, latencies, and frequencies. 
Fast LHC-type devices need to be integrable in the same manner as slow calorimetric or rolling-shutter read-out devices. 

Consisting of six pixel detector planes equipped with small-pitch $\Mimosa$ sensors,
 the mechanics for precise positioning of the device under test (DUT) and the telescope planes in the beam, trigger capabilities, and the $\eudaq$ data acquisition system, 
 the DESY-type beam telescopes meet the requirements in terms of easy integration capabilities, spatial resolution, and trigger rates. 
The telescope planes are designed and built to keep the material budget as low as possible in order to achieve an excellent pointing resolution
 even at the rather low DESY test beam energies of up to 6\,GeV/$\cspeed$.

Several replicas of the DESY-type beam telescope have been built:
 the AIDA telescope residing at the H6, SPS, CERN, the ATLAS copy ACONITE, which will also be operated at H6, ANEMONE at ELSA, University of Bonn, the copy for the Carlton University called CALADIUM, 
 $\Datura$ at DESY-II and, currently under construction, $\Duranta$, which will also be operated at DESY. 
They are all kept alike in order to ease the maintenance. 
The results reported here are based on data taken with the $\Datura$ at test beam\,21 at DESY and with the ACONITE at SLAC and SPS, but are applicable to all other copies. 

The paper is organized as follows. 
The beamlines relevant for the data analysed in this work are introduced in section~\ref{sec:beamlines}, followed by the description of the beam telescope layout
 and the $\eudaq$ framework in sections~\ref{sec:tscope} and~\ref{sec:eudaq}, respectively.
Section~\ref{sec:offline} details the offline analysis and reconstruction flow of the EUTelescope software. 
Finally, in section~\ref{sec:trackres} results of the DESY-type beam telescope performance are presented. % and the pointing resolution for different telescope configurations at various beam momenta is shown. 
The measured data is complemented by analytic predictions using the concept of General Broken Lines (GBL).
%With these GBL predictions at hand, the optimal telescope geometry can be calculated for a given set-up (DUT, cooling, ...) prior to the test beam.
