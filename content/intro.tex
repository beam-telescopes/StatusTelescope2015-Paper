
Beam telescopes are vital tools for R\&D projects focussing on particle detection sensors. 
These range from collider-specific detectors with high radiation tolerance~\cite{1748-0221-9-12-C12001,1748-0221-9-12-C12029},
 high resolution and low mass requirements~\cite{1748-0221-10-03-C03044} to medical applications~\cite{Ballabriga2011S15}, among others. 
Complementary to sensor simulations using finite element analysis tools, test beam studies are used at various stages including the development of either the sensor or the readout chip,
 and subsequently when testing a complete demonstrator with a full data acquisition system. 
%Also in future high-energy physics experiments pixel detectors will be used in the inner layers of the tracking devices. 
%The demands are challenging and range from high speed and high radiation-hardness for the high-luminosity LHC (HL-LHC)
% experiments\,\cite{Nurnberg:2014aya, Garcia-Argos:2015zda} to high resolution and low mass at the International Linear Collider (ILC)\,\cite{ILC} or at the Compact Linear Collider (CLIC)\,\cite{CLIC}. 
Such studies are well suited and often used for the evaluation of the performance of a detector prototype. % developed within the various R\&D projects. 
In order to facilitate these measurements a high-resolution pixel beam telescope was developed within the $\eudet$ project~\cite{ref:eudetreport200902},
 with the clear guideline to allow for an easy integration of user data acquisition systems for a wide range of readout schemes, latencies, and frequencies. 
This is achieved by well defined interfaces on both the hardware and the software level. 
Fast LHC-type tracking devices are integrable in the same manner as slow rolling-shutter readout devices. 

Consisting of six pixel detector planes equipped with fine-pitch $\Mimosa$ sensors~\cite{HuGuo2010480},
 the mechanics for precise positioning of the device under test (DUT) and the telescope planes in the beam, trigger capabilities, and its data acquisition system, 
 the $\DESY$-type beam telescopes meet the requirements in terms of easy integration capabilities, spatial resolution, and trigger rates. 
The telescope planes are designed and built to keep the material budget as low as possible in order to achieve an excellent pointing resolution
 even at the rather low DESY test beam energies of up to 6\,GeV.

The original EUDET beam telescope, which was modified to become the AIDA telescope, is operated at SPS~H6 (CERN).
Responding to the increasing demand of sensor R\&D community, several replicas, here called $\DESY$-type beam telescopes, have been built since then:
 the ATLAS copy ACONITE, which will also be operated at H6, ANEMONE at ELSA (University of Bonn), the copy for the Carlton University called CALADIUM, 
 $\Datura$ at DESY, and --- currently under construction --- $\Duranta$, which will also be operated at DESY. 
All replicas are based on the $\Mimosa$ sensors and are equipped with the same data acquisition system and software framework. 
The EUDET telescope has been used since 2007 in various stages of development by hundreds of users and played an important role in sensor studies employed by a wide community. 
From January 2013 until March 2014 alone, about 300 users utilised a $\DESY$-type beam telescope at DESY for a total of 80 user weeks. 
The results reported here are based on data taken with $\Datura$ at test beam area~21 at {DESY-II} and with ACONITE at SLAC~ESTB and SPS~H6, but are applicable to all other copies. 

The paper is organised as follows: 
The beamlines relevant for the data analysed in this work are introduced in section~\ref{sec:beamlines}, followed by the description of the beam telescope
 and the data acquisition framework in sections~\ref{sec:tscope} and~\ref{sec:eudaq}, respectively.
Section~\ref{sec:offline} details the offline analysis and reconstruction software. 
Results of the $\DESY$-type beam telescope performance are presented in section~\ref{sec:trackres}. % and the pointing resolution for different telescope configurations at various beam momenta is shown. 
The measured data is complemented by analytic predictions using the formalism of General Broken Lines (GBL)~\cite{Blobel20111760,Kleinwort-2012}.
Section~\ref{sec:dutintegration} discusses the integration of DUTs into the beam telescope. 
%With these GBL predictions at hand, the optimal telescope geometry can be calculated for a given set-up (DUT, cooling, ...) prior to the test beam.
