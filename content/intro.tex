
%History of DESY beam telescopes, from Eudet to the present

Also in future high-energy physics experiments pixel detectors will be used in the inner layers of the tracking devices. 
The demands are challenging and range from high speed and high radiation-hardness for the high-luminosity LHC (HL-LHC)
 experiments\,\cite{Nurnberg:2014aya, Garcia-Argos:2015zda} to high resolution and low mass at the International Linear Collider (ILC)\,\cite{ILC} or at the Compact Linear Collider (CLIC)\,\cite{CLIC}. 
Test beam measurements are required in order to evaluate the performance of detector prototypes developed within the various R\&D projects. 
In order to facilitate these measurements a high-resolution pixel beam telescope was developed within the \eudet project.\,\cite{ref:eudetreport200902}
It provides the necessary infrastructure, i.e.~the beam telescope with six planes equipped with precise $\Mimosa$ sensors,
 the mechanics to position the device under test (DUT) in the beam between the two telescope arms, trigger capabilities, and the \eudaq data acquisition system. 
The telescope planes are designed and built to keep the material budget as small as possible in order to achieve an excellent pointing resolution
 even at the rather low DESY test beam energies of less than 6\,GeV/$\cspeed$.

Several replicas of the beam telescope have been built in the meantime, e.g.~the AIDA telescope at CERN as well as the ATLAS copy ACONITE, ANEMONE at Bonn University, CALADIUM at Carlton University, 
 $\Datura$ and, currently under construction, $\Duranta$ at DESY. 
They are all kept alike in order to ease the maintenance. 
The results reported here, refer to data taken with $\Datura$ beam telescope, the device installed in test beam\,21 at DESY, but are applicable to all other copies. 

The paper is organized as follows. 
The beamlines are introduced in section~\ref{sec:beamlines}, followed by the description of the telescope layout and the EUDAQ framework in sections~\ref{sec:tscope} and~\ref{sec:eudaq}, respectively.
Section~\ref{sec:offline} details the offline analysis and reconstruction flow as collected in the EUTelescope software. 
Finally, in section~\ref{sec:trackres} results are presented of the $\Datura$ telescope performance and pointing resolution for different telescope configurations at the DESY test beam. 
Additionally, measurements at SLAC and SPS taken with ACONITE are added and compared to theoretical predictions using the concept of general broken lines (GBL).
With these GBL predictions at hand, the optimal telescope geometry can be calculated for a given set-up (DUT, cooling, ...) prior to the test beam.
