
%Test beam measurements with high precision pixel beam telescopes have proven to be a versatile tool for a wide range of semiconductor sensor studies. 
%Among other beam telescopes, also the $\eudet$-type beam telescopes are used as a key ingredient for many sensors R\&D projects. 

In this work the $\eudet$-type beam telescopes have been described and their performance has been investigated. 
The highly flexible and versatile $\eudet$-type beam telescopes come with a Trigger Logic Unit with time stamping capabilities and a clearly defined interface for an integration of user data acquisition systems.
It is complemented by a data acquisition system for the telescope sensors, and the powerful software packages $\eudaq$ and $\EUTelescope$
 ---  a modular data acquisition framework and an offline reconstruction software, respectively. 
The intrinsic resolution of the $\Mimosa$ sensors have been measured to be $(3.25 \pm 0.03)\,\upmu\meter$.
General Broken Lines calculations predict a track resolution of $1.75\,\upmu\meter$ on the DUT with 20\,mm plane spacing for thin sensors with $\epsdut = 0.001$.

Continuous development efforts are ongoing to upgrade and enhance the Trigger Logic Unit, the data acquisition system, and offline reconstruction framework~\cite{ref:tipp2014_eudaq}.  
The maximum achievable rate will be increased by the successor of the TLU. 
A new plane comprising four Mimosa\,28 planes has been built for the AIDA telescope, covering a detection area of $16\,\centi\meter^2$, and an appropriate DAQ has been developed to cope with the higher data rates. 
%This is only possible with the implementation of parallel data streams for the telescope planes and the DUT, which will be available with \eudaq\,{2.0},
% which will incorporate a new event format supporting multiple triggers per $\Mimosa$ read-out frame. 
Future versions of \eudaq\ will provide support for asynchronous data streams allowing to operate certain devices at a much higher trigger frequency than others
 and thus making full use of the the beams available in the various beam lines.
The development of the $\EUTelescope$ analysis software is driven by the large and diverse user community. 
Features currently under development include tracking in magnetic fields and more accurate detector geometry descriptions. 
