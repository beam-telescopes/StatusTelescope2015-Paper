
%Test beam measurements with high precision pixel beam telescopes have proven to be a versatile tool for a wide range of semiconductor sensor studies. 
%Among other beam telescopes, also the DESY-type beam telescopes are used as a key ingredient for many sensors R\&D projects. 

The highly flexible and versatile DESY-type beam telescopes come with a complete data acquisition system, a Trigger Logic Unit with time stamping capabilities
 and a clearly defined interface for an integration of user data acquisition systems.
It also provides the powerful software packages $\eudaq$, a modular data acquisition framework, and $\EUTelescope$ for data analysis. 
% to intro X users used a DESY-type beam telescopes only in 2013 for a total of n weeks alone at the DESY-II test beam areas. 
This work the performance of the  DESY-type beam telescopes has been investigated. 
At 5\,GeV beam energy, a measured unbiased residual width of $(3.96 \pm 0.03)\,\upmu\meter$ is achievable
 and the intrinsic resolution of the $\Mimosa$ sensors have been measured to $(3.43 \pm 0.03)\,\upmu\meter$.
General Broken Lines calculations predict a pointing resolution of $1.89\,\upmu\meter$ on the DUT in a \textit{five planes plus DUT} configuration with 20\,mm plane spacing,
 which well agrees wit a measured pointing resolution of $(1.94 \pm 0.08)\,\upmu\meter$.

A continuous development is ongoing to upgrade and enhance the data acquisition system itself, and also the data acquisition and analysis framework.  
The maximum achievable rate will be increased by the successor of the TLU, allowing for a trigger rate of up to 1\,MHz.
%This is only possible with the implementation of parallel data streams for the telescope planes and the DUT, which will be available with \eudaq\,{2.0},
% which will incorporate a new event format supporting multiple triggers per $\Mimosa$ read-out frame. 
 A new data format will support multiple triggers per mimosa read-out frame facilitating asynchronous data acquisition. 
The development of the $\EUTelescope$ analysis software is driven by the large and diverse user community. 
Features currently under development include tracking in magnetic fields and more accurate detector geometry descriptions. 

