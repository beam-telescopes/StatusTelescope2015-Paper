% Test beam in general and at DESY
Test beam facilities have shown up as working horses for future detector development for LHC upgrades as well as the future linear colliders.
At the site in Hamburg, Germany, DESY operates the electron/positron synchrotron DESY II, which is mainly an injector for DORIS or PETRA, however, which also incorporates three test beam areas. 
The circumference of the DESY II ring is 292.8\,m. 
It accelerates and decelerates in a sinusoidal mode with a frequency of 12.5\,Hz. 
Therefore, one DESY II cycle takes 80\,ms, the bunch length is around 30\,ps. 

% Principle of Test beam generation 
Test beams for detector development are generated by bremsstrahlung in a carbon fibre first. 
Second, the resulting bremsstrahlung photons are converted to electron/positron pairs with a metal plate. 
Using a dipole magnet, this secondary electron/positron beam is spread out.
The final test beam is cut out with a collimator. 
Thus, beam momenta between 1 up to 6\,GeV/c with a rate $\sim 1$\,kHz can be adjusted. A maximum rate of $> 9$ kHz around 2.5\,GeV can be reached in optimum. A detailed description of the test beams at DESY II is given in \cite{EUDET-2007-11}.

% CERN and SLAC
In addition, our beam telescopes were also characterised at other test beam facilities: 
At CERN, Geneva, Switzerland, the Super Proton Synchrotron (SPS) provides a secondary pion beam with momenta between 5 to 205\,GeV/c. 
One beam spill contains $2\cdot10^8$\,particles and is 4.8 to 9.6\,s long, the spill rate is every 14 to 48\,s \cite{???}.  
% Talks when google: SPS CERN test beam
% 5 – 205 GeV/c
% Particle intensity max. 2*108 particles per spill
% spill length4.8 to 9.6 s 
% spill every 14s – ~48s
% more at: http://ps-schedule.web.cern.ch/ps-schedule/
At SLAC, Menlo Park, US, the End Station Test Beam (ESTB) provides a secondary electron beam with a momenta between 2 to 15\,GeV/c.
A beam pulse incorporates typically 109\,particles at 5~Hz \cite{d}.
% https://portal.slac.stanford.edu/sites/ard_public/facet/newnav/Pages/tf/estb/ESTB-Beam.aspx

Results on the resolution of the beam telescopes depending on different test beams and, thus, beam energies are described in section~\ref{sec:trackres}. 





