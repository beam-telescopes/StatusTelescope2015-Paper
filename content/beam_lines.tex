Test beam facilities have shown up as working horses for future detector development for LHC upgrades as well as the future linear colliders.
At the site in Hamburg, DESY operates the electron/positron synchrotron DESY II, which is mainly an injector for DORIS or PETRA, however, which also incorporates three test beam areas. 
The circumference of the DESY II ring is 292.8 m. 
It accelerates and decelerates in a sinusoidal mode with a frequency of 12.5 Hz. 
Therefore, one DESY II cycle takes 80 ms, the bunch length is around 30 ps. 

Test beams for detector development are generated by bremsstrahlung in a carbon fibre first. 
Second, the resulting bremsstrahlung photons are converted to electron/positron pairs with a metal plate. 
Using a dipole magnet, this beam is spread out.
The final test beam is cut out with a collimator. 
Thus, beam momenta between 1 and 5 GeV with a rate $> 1$ kHz can be adjusted. iDepending on the total, a maximum rate of $> 9$ kHz between 2-3 GeV can be reached in optimum.    

A detailed description of the test beams at DESY II is given in the EUDET-Memo-2007-11 ``Test Beams at DESY''. 


SPS at CERN.

ESTB (?) at SLAC.




