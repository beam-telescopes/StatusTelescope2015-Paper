
For this work, data has been taken at three different test beam facilities, namely at {DESY-II}, at SPS, H6 and at SLAC, ETSB.
% Test beam in general and at DESY
%Test beam facilities have shown up as working horses for detector development for LHC upgrades as well as the future linear colliders.
The DESY-II electron/positron synchrotron at the DESY site in Hamburg has a circumference of 292.8\,m and is mainly used as an injector for the PETRA-III storage ring. 
However, it also supplies beam to three test beam areas, one  of them being the test beam area\,21.
%The circumference of the DESY-II ring is 292.8\,m. 
Its dipole magnets operate in a sinusoidal ramping mode with a frequency of 12.5\,Hz. 
Therefore, one DESY-II cycle takes 80\,ms, the bunch length is around 30\,ps. 

% Principle of Test beam generation 
The DESY-II synchrotron is equipped with movable carbon fibres. 
When these are positioned in the beam, bremsstrahlung photons are created leaving the beam line tangentially.
Subsequently, the bremsstrahlung photons are converted to electron/positron pairs with a copper plate. 
Their energy distribution reaches up to 6\,GeV. 
Using a dipole magnet, this secondary electron/positron beam is spread out.
A collimator can be used to select certain energy ranges of the beams reaching the experimental halls with an achievable rate of up to $\sim 10$\,kHz. 
%In optimal conditions, a maximal particle rate of $9$\,kHz at around 2.5\,GeV/$\cspeed$ can be reached. 
A more detailed description of the test beams at {DESY-II} is given in reference~\cite{EUDET-2007-11}.

% CERN and SLAC
%In addition to measurements at DESY\,II, the beam telescopes were also characterised at other test beam facilities: 
At CERN the circular Super Proton Synchrotron (SPS) of 6.9\,km length is the final injector of proton beams for the Large Hadron Collider. 
Additionally, it provides a proton beam on three secondary targets resulting in a mixed beam (electrons, hadrons, muons) with energies between 5\,GeV to 205\,GeV
 feeding up to seven beam lines in the SPS North Area, one of them being the H6 beam line. 
One beam spill contains up to $\sim 2\cdot10^8$\,particles and is 4.8\,s to 9.6\,s long. 
A spill occurs every 14\,s to 48\,s.\,\cite{SPS} 
% Talks when google: SPS CERN test beam
% 5 – 205 GeV/c
% Particle intensity max. 2*108 particles per spill
% spill length4.8 to 9.6 s 
% spill every 14s – ~48s
% more at: http://ps-schedule.web.cern.ch/ps-schedule/

At SLAC the Linac Coherent Light Source incorporates a primary electron beam with up to 13.6\,GeV and a repetition rate of 120\,Hz.
Either this primary beam, or a secondary beam produced on a thin target can be sent to the End Station Test Beam (ESTB) with energies between 3.5\,GeV and 13.6\,GeV at 5\,Hz
 resulting in up to $10^{9}$ electrons per pulse.\,\cite{SLAC}

%A beam pulse incorporates typically 109\,particles at 5\,Hz.\,\cite{d}
% https://portal.slac.stanford.edu/sites/ard_public/facet/newnav/Pages/tf/estb/ESTB-Beam.aspx

%Results on the resolution of the beam telescopes depending on different test beams and, thus, beam energies are described in section~\ref{sec:trackres}. 





