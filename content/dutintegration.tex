%mechanic
A DUT can be mechanically integrated into the $\eudet$-type beam telescopes at three positions: upstream or downstream of the telescope, or between the two telescope arms.
If placed between the arms, micrometer precision $xy\phi$-stages are available for translation scans and rotation of the DUT~\cite{Mimosa-twiki}.
At this position, the maximum width of the DUT set-up is 500\,mm.
DUTs with larger spatial dimensions are therefore placed at the rear end of the beam telescope. 

%trigger and daq sw
User DAQ systems are integrated to the TLU the same way as the beam telescope DAQ itself using either the RJ45 or the LEMO interface, cf.~section~\ref{sec:tdaq}.
The handshake mode is configurable for each integrated systems individually. 
For the integration of the DUT data stream with EUDAQ, a producer capable of receiving commands by the Run Control and sending data to the $\eudaq$ Data Collector is necessary
 as described in section~\ref{sec:eudaq}.

%eutelescope
A dedicated alignment run prior to data taking with no DUT in the beam allows for precise alignment of the telescope planes, especially for larger $\epsdut$. 
With a proper alignment at hand, runs including a DUT are to be analysed subsequently in order to align the DUT with respect to the beam telescope. 
The reconstructed tracks can then be used to characterise the DUT, i.e.~to measure its intrinsic resolution or efficiency. 

With increasing $\epsdut$, and thus multiple scattering within the DUT, the choice of the tracking algorithm needs further consideration. 
In general, a GBL fit produces tracks with a lower $\chi^2$ compared to straight line fits,
 as kinks at the possibly thick DUT and also at the $\Mimosa$ planes themselves are allowed for.
Therefore, using GBL for track fitting is recommended. 
% The inclusion of downstream planes using straight line fits might significantly deteriorate the resolution,
%If straight line fitting is used nevertheless, a fit using only the upstream planes might result in a narrower unbiased residual width, compared to a fit that includes the downstream planes. 
%This is due to a bias of the extrapolated tracks at the DUT by the downstream planes after traversing material. 

For comparison, in a narrow configuration using a 6\,GeV electron/positron beam,
 the track resolution of the EUDET-type beam telescope at the DUT using GBL is $1.83\,\upmu\meter$ making use of all six planes. 
The optimal configuration using only the upstream planes is a wide configuration with $\dz = 150\,\milli\meter$,
 and the achievable track resolution is at best $3.9\,\upmu\meter$. 
Hence, using only the upstream planes deteriorates the resolution by more than $2\,\upmu\meter$.
